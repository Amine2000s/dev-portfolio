\documentclass[10pt, letterpaper]{article}

% Packages:
\usepackage[
    ignoreheadfoot, % set margins without considering header and footer
    top=2 cm, % seperation between body and page edge from the top
    bottom=2 cm, % seperation between body and page edge from the bottom
    left=2 cm, % seperation between body and page edge from the left
    right=2 cm, % seperation between body and page edge from the right
    footskip=1.0 cm, % seperation between body and footer
    % showframe % for debugging 
]{geometry} % for adjusting page geometry
\usepackage{titlesec} % for customizing section titles
\usepackage{tabularx} % for making tables with fixed width columns
\usepackage{array} % tabularx requires this
\usepackage[dvipsnames]{xcolor} % for coloring text
\definecolor{primaryColor}{RGB}{0, 79, 144} % define primary color
\usepackage{enumitem} % for customizing lists
\usepackage{fontawesome5} % for using icons
\usepackage{amsmath} % for math
\usepackage[
    pdftitle={CHABI Amine sif eddine  CV},
    pdfauthor={CHABI Amine sif eddine},
    pdfcreator={LaTeX with RenderCV},
    colorlinks=true,
    urlcolor=primaryColor
]{hyperref} % for links, metadata and bookmarks
\usepackage[pscoord]{eso-pic} % for floating text on the page
\usepackage{calc} % for calculating lengths
\usepackage{bookmark} % for bookmarks
\usepackage{lastpage} % for getting the total number of pages
\usepackage{changepage} % for one column entries (adjustwidth environment)
\usepackage{paracol} % for two and three column entries
\usepackage{ifthen} % for conditional statements
\usepackage{needspace} % for avoiding page brake right after the section title
\usepackage{iftex} % check if engine is pdflatex, xetex or luatex

% Ensure that generate pdf is machine readable/ATS parsable:
\ifPDFTeX
    \input{glyphtounicode}
    \pdfgentounicode=1
    % \usepackage[T1]{fontenc} % this breaks sb2nov
    \usepackage[utf8]{inputenc}
    \usepackage{lmodern}
\fi



% Some settings:
\AtBeginEnvironment{adjustwidth}{\partopsep0pt} % remove space before adjustwidth environment
\pagestyle{empty} % no header or footer
\setcounter{secnumdepth}{0} % no section numbering
\setlength{\parindent}{0pt} % no indentation
\setlength{\topskip}{0pt} % no top skip
\setlength{\columnsep}{0cm} % set column seperation
\makeatletter
\let\ps@customFooterStyle\ps@plain % Copy the plain style to customFooterStyle
\patchcmd{\ps@customFooterStyle}{\thepage}{
    \color{gray}\textit{\small CHABI Amine sif eddine - Page \thepage{} sur \pageref*{LastPage}}
}{}{} % replace number by desired string
\makeatother
\pagestyle{customFooterStyle}

\titleformat{\section}{\needspace{4\baselineskip}\bfseries\large}{}{0pt}{}[\vspace{1pt}\titlerule]

\titlespacing{\section}{
    % left space:
    -1pt
}{
    % top space:
    0.3 cm
}{
    % bottom space:
    0.2 cm
} % section title spacing

\renewcommand\labelitemi{$\circ$} % custom bullet points
\newenvironment{highlights}{
    \begin{itemize}[
        topsep=0.10 cm,
        parsep=0.10 cm,
        partopsep=0pt,
        itemsep=0pt,
        leftmargin=0.4 cm + 10pt
    ]
}{
    \end{itemize}
} % new environment for highlights

\newenvironment{highlightsforbulletentries}{
    \begin{itemize}[
        topsep=0.10 cm,
        parsep=0.10 cm,
        partopsep=0pt,
        itemsep=0pt,
        leftmargin=10pt
    ]
}{
    \end{itemize}
} % new environment for highlights for bullet entries


\newenvironment{onecolentry}{
    \begin{adjustwidth}{
        0.2 cm + 0.00001 cm
    }{
        0.2 cm + 0.00001 cm
    }
}{
    \end{adjustwidth}
} % new environment for one column entries

\newenvironment{twocolentry}[2][]{
    \onecolentry
    \def\secondColumn{#2}
    \setcolumnwidth{\fill, 4.5 cm}
    \begin{paracol}{2}
}{
    \switchcolumn \raggedleft \secondColumn
    \end{paracol}
    \endonecolentry
} % new environment for two column entries

\newenvironment{header}{
    \setlength{\topsep}{0pt}\par\kern\topsep\centering\linespread{1.5}
}{
    \par\kern\topsep
} % new environment for the header

\newcommand{\placelastupdatedtext}{% \placetextbox{<horizontal pos>}{<vertical pos>}{<stuff>}
  \AddToShipoutPictureFG*{% Add <stuff> to current page foreground
    \put(
        \LenToUnit{\paperwidth-2 cm-0.2 cm+0.05cm},
        \LenToUnit{\paperheight-1.0 cm}
    ){\vtop{{\null}\makebox[0pt][c]{
        \small\color{gray}\textit{Dernière mise à jour en juillet 2025}\hspace{\widthof{Dernière mise à jour en juillet 2025}}
    }}}%
  }%
}%

% save the original href command in a new command:
\let\hrefWithoutArrow\href

% new command for external links:
\renewcommand{\href}[2]{\hrefWithoutArrow{#1}{\ifthenelse{\equal{#2}{}}{ }{#2 }\raisebox{.15ex}{\footnotesize \faExternalLink*}}}


\begin{document}
    \newcommand{\AND}{\unskip
        \cleaders\copy\ANDbox\hskip\wd\ANDbox
        \ignorespaces
    }
    \newsavebox\ANDbox
    \sbox\ANDbox{}

    \placelastupdatedtext
    \begin{header}
    \textbf{\fontsize{24 pt}{24 pt}\selectfont Amine Sif Eddine Chabi}

    \vspace{0.3 cm}

    \normalsize
    \mbox{{\color{black}\footnotesize\faMapMarker*}\hspace*{0.13cm}Algérie – Ouvert à la mobilité}%
    \kern 0.25 cm%
    \AND%
    \mbox{\hrefWithoutArrow{mailto:chabiaminesifeddine@gmail.com}{\color{black}{\footnotesize\faEnvelope[regular]}\hspace*{0.13cm}chabiaminesifeddine@gmail.com}}%
    \kern 0.25 cm%
    \AND%
    \mbox{\hrefWithoutArrow{tel:+213777157757}{\color{black}{\footnotesize\faPhone*}\hspace*{0.13cm}+213 7 77 15 77 57}}%
    \kern 0.25 cm%
    \AND%
    \mbox{\hrefWithoutArrow{https://amine2000s.github.io/personal-website/}{\color{black}{\footnotesize\faLink}\hspace*{0.13cm}amine2000s.github.io}}%
    \kern 0.25 cm%
    \AND%
    \mbox{\hrefWithoutArrow{https://linkedin.com/in/amine-chabi-a90b7822a}{\color{black}{\footnotesize\faLinkedinIn}\hspace*{0.13cm}amine-chabi-a90b7822a}}%
    \kern 0.25 cm%
    \AND%
    \mbox{\hrefWithoutArrow{https://github.com/Amine2000s}{\color{black}{\footnotesize\faGithub}\hspace*{0.13cm}Amine2000s}}%
\end{header}

    \vspace{0.3 cm - 0.3 cm}


    \section{Profil}

\begin{onecolentry}
Ingénieur DevOps avec une approche d'ingénierie logicielle, alliant expérience en développement backend et expertise en automatisation d'infrastructure. Maîtrise de Docker, CI/CD et déploiements sur VPS, avec une expérience pratique en Spring Boot, Express.js, MySQL, Redis et scripts Python. Passionné par la construction de systèmes évolutifs et fiables grâce à l'automatisation et aux opérations pilotées par le code.\end{onecolentry}


    
    %\section{Quick Guide}

    %\begin{onecolentry}
     %   \begin{highlightsforbulletentries}


      %  \item Each section title is arbitrary and each section contains a list of entries.

       % \item There are 7 unique entry types: \textit{BulletEntry}, \textit{TextEntry}, \textit{EducationEntry}, \textit{ExperienceEntry}, \textit{NormalEntry}, \textit{PublicationEntry}, and \textit{OneLineEntry}.

        %\item Select a section title, pick an entry type, and start writing your section!

        %\item \href{https://docs.rendercv.com/user_guide/}{Here}, you can find a comprehensive user guide for RenderCV.


        %\end{highlightsforbulletentries}
    %\end{onecolentry}

    \section{Formation}

\begin{twocolentry}{
    \textit{Sep 2021 – Jul 2024}}
    \textbf{Université de Biskra}

    \textit{Licence en Informatique – Systèmes Informatiques}
\end{twocolentry}

\vspace{0.10 cm}
\begin{onecolentry}
    \begin{highlights}
        \item Formation couvrant les domaines clés : algorithmes, structures de données, systèmes d'exploitation, compilateurs, logique, algèbre, cybersécurité, réseaux, intelligence artificielle, développement mobile et web
        \item Bénévolat au Symposium International sur les Nouvelles Technologies de l'Information et de la Communication (ISNIB), contribution à la logistique événementielle et à la coordination académique
        \item Participation aux clubs technologiques étudiants et soutien aux activités de sensibilisation académique aux côtés du département d'informatique
        \item Focus sur l'architecture système et le développement backend
        \item Cours incluant le développement Java, les systèmes de bases de données, la modélisation UML

    \end{highlights}
\end{onecolentry}

\section{Expérience}

\begin{twocolentry}{
    \textit{Oct 2025 – Présent}}
    \textbf{Ingénieur DevOps / NetDevOps}
    
    \textit{Infraxcode – Temps plein}
\end{twocolentry}

\vspace{0.10 cm}
\begin{onecolentry}
    \begin{highlights}
        \item Maintenance et amélioration de l'infrastructure sur site, en se concentrant sur l'automatisation, la scalabilité et la fiabilité
        \item Prototypage et gestion d'environnements basés sur AWS pour les tests et les déploiements de preuve de concept
        \item Développement de scripts d'automatisation (Python, Bash) pour rationaliser le provisionnement, la surveillance et les opérations réseau
        \item Construction et amélioration des pipelines CI/CD utilisant GitLab CI pour soutenir les workflows de développement et de déploiement
        \item Collaboration inter-équipes pour intégrer les pratiques d'automatisation réseau (NetDevOps) dans les processus DevOps
    \end{highlights}
\end{onecolentry}

\vspace{0.2 cm}

\begin{twocolentry}{
    \textit{Temps partiel · Août 2025 – Présent}}
    \textbf{Consultant Junior en Transformation Digitale}
    
    \textit{GIZ – Deutsche Gesellschaft für Internationale Zusammenarbeit}
\end{twocolentry}

\vspace{0.10 cm}
\begin{onecolentry}
    \begin{highlights}
        \item Accompagnement des petites et moyennes entreprises (PME) algériennes dans leurs initiatives de transformation digitale
        \item Réalisation d'évaluations des besoins et conseil sur l'adoption d'outils numériques, de services cloud et d'automatisation des processus
        \item Collaboration avec des équipes pluridisciplinaires pour concevoir des feuilles de route pour la modernisation IT et l'optimisation des processus
        \item Contribution à des ateliers promouvant la littératie numérique et l'intégration durable de la technologie pour les PME
    \end{highlights}
\end{onecolentry}

\vspace{0.2 cm}


\begin{twocolentry}{
    \textit{Biskra, Algérie}    
        
    \textit{Jan 2025 – Avril 2025}}
    \textbf{Développeur Back-End \& Chef d'Équipe}
    
    \textit{Cellule Qualité – Université de Biskra}
\end{twocolentry}

\vspace{0.10 cm}
\begin{onecolentry}
    \begin{highlights}
        \item Architecture d'un backend Spring Boot en couches pour un système de gestion de la qualité
        \item Développement d'API REST sécurisées et conformes HATEOAS avec contrôle d'accès basé sur les rôles utilisant Spring Security
        \item Intégration de la compression GZIP, Spring AOP et limitation du débit pour optimiser les performances
        \item Implémentation d'un logging asynchrone centralisé avec MDC pour une traçabilité avancée
        \item Direction d'une équipe agile, définition de sprints hebdomadaires, suivi des tâches quotidiennes et rédaction de documentation technique
        \item \textbf{Technologies:} Java 17, Spring Boot, Spring Security, MySQL, Docker, Maven, Git, Swagger
    \end{highlights}
\end{onecolentry}

\vspace{0.2 cm}

\begin{twocolentry}{
    \textit{Distanciel / Algérie}    
        
    \textit{Jul 2024 – Nov 2024}}
    \textbf{Développeur Back-End}
    
    \textit{Algerian Tech Makers}
\end{twocolentry}

\vspace{0.10 cm}
\begin{onecolentry}
    \begin{highlights}
        \item Contribution au développement d'une plateforme RH interne
        \item Conception de schémas de bases de données relationnelles et sécurisation d'API REST utilisant l'authentification par token
        \item \textbf{Technologies:} TypeScript, Express.js, PostgreSQL
    \end{highlights}
\end{onecolentry}




    
 




    
    %\section{Publications}



        
        %\begin{samepage}
        %    \begin{twocolentry}{
        %        Jan 2004
        %    }
        %        \textbf{3D Finite Element Analysis of No-Insulation Coils}

         %       \vspace{0.10 cm}
%
         %       \mbox{Frodo Baggins}, \mbox{\textbf{\textit{John Doe}}}, \mbox{Samwise Gamgee}
        %    \end{twocolentry}


       %     \vspace{0.10 cm}
%
       %     \begin{onecolentry}
       % \href{https://doi.org/10.1109/TASC.2023.3340648}{10.1109/TASC.2023.3340648}
      %      \end{onecolentry}
    %    \end{samepage}

\section{Leadership et Implication Communautaire}

\begin{twocolentry}{
    \textit{Avr 2023 – Jul 2024}}
    \textbf{Microsoft Learn Student Ambassador – Rang Beta}
    
    \textit{Microsoft Learn}
\end{twocolentry}

\vspace{0.10 cm}
\begin{onecolentry}
    \begin{highlights}
        \item Animation d'ateliers sur Git et GitHub et promotion des meilleures pratiques de collaboration
    \end{highlights}
\end{onecolentry}

\vspace{0.2 cm}

\begin{twocolentry}{
    \textit{Sep 2022 – Jul 2024}}
    \textbf{Membre Fondateur – Club Scientifique Debug}
    
    \textit{Université de Biskra}
\end{twocolentry}

\vspace{0.10 cm}
\begin{onecolentry}
    \begin{highlights}
        \item Co-fondation d'un club technologique dirigé par les étudiants, axé sur l'apprentissage par les pairs, la culture de la programmation et la collaboration interdisciplinaire
        \item Organisation d'événements réguliers incluant des ateliers de codage, des panels de conférenciers invités et des présentations de projets logiciels
        \item Accompagnement des étudiants de première année avec du mentorat académique et encadrement de mini-projets de groupe
        \item Coordination de la communication avec le département d'informatique et contribution à la croissance de la participation du club à plus de 80 membres actifs
    \end{highlights}
\end{onecolentry}

    
 \section{Projets}
 
\begin{twocolentry}{
    \textit{}}
    \textbf{Scraper Web MaharaTech}
    \textit{\href{https://amine2000s.github.io/personal-website/blog/bac-web-scraper/}{\texttt{Lire l'article}}}}
\end{twocolentry}

\vspace{0.10 cm}

\begin{onecolentry}
    \begin{highlights}
        \item Développement d'un scraper basé sur Python pour préserver le contenu des cours MaharaTech pendant une coupure internet nationale
        \item Utilisation de Playwright pour l'automatisation de navigateur headless et contournement de connexion via cookies de session
        \item Exploration du DOM pour extraire la structure des cours, liens YouTube intégrés dans des iframes dynamiques
        \item Parsing et sauvegarde des leçons, liens vidéo et métadonnées pour un accès hors ligne
        \item Technologies: Python, Playwright, BeautifulSoup, HTML/JS, Injection de Cookies
    \end{highlights}
\end{onecolentry}
  %\  \begin{twocolentry}{
    %\\textit{In Progress}}
    %\\textbf{SAP BTP App for User Provisioning}
%\\end{twocolentry}
%\
%\\vspace{0.10 cm}
%\\begin{onecolentry}
    %\\begin{highlights}
        %\\item Built a secure Java-based app for provisioning user %\identities via SAP IAS/IPS
        %\\item Integrated with SAP BTP using XSUAA and OAuth2 for %\secure service access
        %\\item Demonstrated full user lifecycle flow: creation, %\update, and deprovisioning
        %\\item \textbf{Concepts:} Identity Access Mgmt, SAP Cloud %\Security, User Federation
 %\   \end{highlights}
%\\end{onecolentry}
%\\vspace{0.10 cm}
%\
\vspace{0.10 cm}

%\\begin{twocolentry}{
  %\  \textit{In Progress}}
    %\\textbf{Vehicle Order Tracking Microservice via S/4HANA}
%\\end{twocolentry}
%\\vspace{0.10 cm}

%\\vspace{0.10 cm}
%\\begin{onecolentry}
  %\  \begin{highlights}
    %\    \item Developed a Java microservice consuming SAP S/4HANA APIs for tracking vehicle order statuses
      %\  \item Connected via SAP API Hub and implemented OAuth2-secured data flow
        %\\item Simulated logistics integration for vehicle production visibility
        %\\item \textbf{Concepts:} SAP OData, S/4HANA Logistics, Java Integration, OAuth2
    %\\end{highlights}
%\\end{onecolentry}
%\\vspace{0.10 cm}




 
\vspace{0.10 cm}
\begin{twocolentry}{
    \textit{}}
    \textbf{Shortify – Raccourcisseur d'URL}
    \textit{\href{https://shortify-url-shortner-584s.onrender.com}{\texttt{shortify.onrender.com}}}}
\end{twocolentry}

\vspace{0.10 cm}


\vspace{0.10 cm}
\begin{onecolentry}
    \begin{highlights}
        \item Développement d'une application Spring Boot en couches pour le raccourcissement d'URL avec tableau de bord d'analytique
        \item Dockerisation et déploiement sur Render (\faExclamationTriangle\ peut avoir un délai de démarrage dû à l'hébergement gratuit)
        \item Base de données MySQL hébergée sur Aiven.io
        \item Technologies: Java, Spring Boot, Thymeleaf, Docker, MySQL
    \end{highlights}
\end{onecolentry}



\vspace{0.2 cm}

\begin{twocolentry}{
    \textit{Proposition GSoC}}
    \textbf{Diomede – Plateforme d'Imagerie Médicale}
\end{twocolentry}

\vspace{0.10 cm}
\begin{onecolentry}
    \begin{highlights}
        \item Proposition et prototypage d'une plateforme d'albums DICOM avec accès sécurisé aux images et outils de métadonnées
        \item Construction d'un PoC JavaFX et initiation d'une réécriture full-stack utilisant Spring Boot et React
        \item \textbf{Technologies:} Spring Boot, MongoDB, React, JavaFX, Docker, Keycloak, DICOM
        \item \href{https://github.com/Amine2000s/dicom-albums-manager}{Dépôt GitHub}
    \end{highlights}
\end{onecolentry}

\vspace{0.2 cm}

\begin{twocolentry}{
    \textit{Biskra, Algérie}}
    \textbf{Système de Gestion des Déchets – Biskra Nadifa}
\end{twocolentry}

\vspace{0.10 cm}
\begin{onecolentry}
    \begin{highlights}
        \item Conception d'un système backend RESTful pour coordonner la collecte des déchets municipaux et la répartition des conducteurs
        \item Développement de schémas de bases de données optimisés et logique de service pour le suivi des tâches en temps réel
        \item \textbf{Technologies:} Java, Spring Boot, MySQL
    \end{highlights}
\end{onecolentry}

\vspace{0.2 cm}

\begin{twocolentry}{
    \textit{Projet GitHub}}
    \textbf{Application de Prévisions Météorologiques avec Cache Redis}
\end{twocolentry}

\vspace{0.10 cm}
\begin{onecolentry}
    \begin{highlights}
        \item Développement d'une application Node.js Express pour afficher les prévisions météorologiques avec recherche dynamique
        \item Utilisation de Redis pour mettre en cache les appels API et réduire la charge sur le fournisseur de services météorologiques
        \item \textbf{Technologies:} Node.js, Express.js, Redis, API OpenWeatherMap
    \end{highlights}
\end{onecolentry}

\vspace{0.2 cm}

\begin{twocolentry}{
    \textit{2023 – Université de Biskra}}
    \textbf{KCube – Contrôleur Android pour Robot de Livraison}
\end{twocolentry}
\begin{onecolentry}
    \begin{highlights}
        \item Développement d'une application mobile pour contrôler un robot Arduino compatible Bluetooth (HC-05)
        \item Intégration du mouvement directionnel, déverrouillage du couvercle et suivi en temps réel basé sur une carte
        \item \textbf{Tech:} Android SDK, Java, API Google Maps, Sockets Bluetooth, Arduino
        \item GitHub: \href{https://github.com/Amine2000s/KCube-Controller-App}{github.com/Amine2000s/KCube-Controller-App}
    \end{highlights}
\end{onecolentry}

\vspace{0.2 cm}

\begin{twocolentry}{
    \textit{Travail universitaire}}
    \textbf{Application Bureau de Gestion de Tâches}
\end{twocolentry}

\vspace{0.10 cm}
\begin{onecolentry}
    \begin{highlights}
        \item Développement d'un gestionnaire de tâches basé sur JavaFX avec stockage persistant MySQL
        \item Implémentation de fonctionnalités telles que les alertes d'échéance et les opérations CRUD avec structure MVC
        \item \textbf{Technologies:} Java, JavaFX, MySQL
    \end{highlights}
\end{onecolentry}


\section{Contributions Open Source}

\begin{twocolentry}{
    \textit{Social Winter of Code 2022}}
    \textbf{Mingler – Plateforme de Réseaux Sociaux}
\end{twocolentry}

\vspace{0.10 cm}
\begin{onecolentry}
    \begin{highlights}
        \item Amélioration de l'expérience utilisateur en ajoutant une modale de confirmation de déconnexion et correction de la réactivité de la mise en page
        \item Reproduction et documentation claire des problèmes, conduisant à des corrections rapides par les mainteneurs
        \item Participation aux revues de PR, ouverture d'issues et contribution active aux discussions communautaires
        \item \textbf{Technologies:} PHP, JavaScript, Bootstrap
        \item \href{https://github.com/yourusername/mingler}{Dépôt GitHub}
    \end{highlights}
\end{onecolentry}

\vspace{0.2 cm}

\begin{twocolentry}{
    \textit{Proposition Google Summer of Code}}
    \textbf{Diomede – Plateforme d'Imagerie Médicale}
\end{twocolentry}

\vspace{0.10 cm}
\begin{onecolentry}
    \begin{highlights}
        \item Proposition d'améliorations architecturales et logique d'extraction de métadonnées DICOM
        \item Réorganisation de la structure du dépôt et amélioration de la documentation pour les contributeurs
        \item Fourniture de conseils dans les discussions d'issues et contribution de pull requests au dépôt principal
        \item \textbf{Technologies:} JavaFX, React, Spring Boot, MongoDB
        \item \href{https://docs.google.com/document/d/1h_PNk40tztVvHmK8rwr_4W8jZD0EXsMA/edit?usp=sharing&ouid=103237394037038598452&rtpof=true&sd=true}{Proposition GSoC}
    \end{highlights}
\end{onecolentry}

\section{Certifications et Formation}

\begin{twocolentry}{
    \textit{Fév 2025 – Mai 2025}}
    \textbf{Programme GIZ – Transformation Digitale et Verte}
    
    \textit{GIZ Algérie (Deutsche Gesellschaft für Internationale Zusammenarbeit)}
\end{twocolentry}

\vspace{0.10 cm}
\begin{onecolentry}
    \begin{highlights}
        \item Sélectionné parmi les meilleurs candidats nationaux pour rejoindre ce programme de formation compétitif et à fort impact
        \item GIZ est l'agence allemande de développement international promouvant la réforme économique et digitale durable dans le monde entier
        \item Parcours certifié complété, qualifiant pour le rôle de Consultant Junior en Transformation Digitale
        \item Expérience pratique acquise dans:
        \begin{itemize}
            \item Gestion des Processus Métier (BPM) avec Camunda
            \item Simulation CRM via la plateforme CAFYB
            \item Analyse de données et création de tableaux de bord avec Microsoft Power BI
        \end{itemize}
        \item Score final: 76,78/100 — classé 3ème sur 30 participants
    \end{highlights}
\end{onecolentry}


\vspace{0.2 cm}

\begin{twocolentry}{
    \textit{}}
    \textbf{Certifications Supplémentaires}
\end{twocolentry}

\vspace{0.10 cm}
\begin{onecolentry}
    \begin{highlights}
        \item Technologie du Jumeau Numérique : Concepts et Applications - Mahara-tech
        \item Fondamentaux de l'Apprentissage Profond – NVIDIA
        \item Spring Boot 2.0 Essential Training – LinkedIn Learning
        \item Docker de Zéro à Héros – Udemy
        \item Fondamentaux API Étudiant – Microsoft Learn
        \item Programme Consultant en Transformation Digitale – GIZ Algérie
    \end{highlights}
\end{onecolentry}

\section{Recommandations}

\begin{twocolentry}{
    \textit{Référence Académique}}
    \textbf{Prof. Kahloul Laid}
    
    \textit{Professeur académique pendant les études de licence - Chef de la Cellule d'assurance qualité, Université de Biskra}
\end{twocolentry}

\vspace{0.10 cm}
\begin{onecolentry}
    \begin{highlights}
        \item Fourni une lettre d'affectation de 2 ans soutenant mon rôle dans le Programme d'Assurance Qualité de l'université
        \item Email: \href{mailto:kahloul.laid@example.edu}{l.kahloul@univ-biskra.dz}
        \item Lettre (Français): \href{https://drive.google.com/drive/u/0/folders/1c4Ir-YR1PDycAvle480V_S6aWhYIowHr}{Google Drive}
        \item Lettre d'affectation (Français): \href{https://drive.google.com/file/d/1l2ttOWNcUNGIsGYwFtOykfdDQigjMvBm/view?usp=sharing}{Google Drive}
    \end{highlights}
\end{onecolentry}

\vspace{0.2 cm}

\begin{twocolentry}{
    \textit{Référence Académique}}
    \textbf{M. Ramdhani}
    
    \textit{Superviseur du projet de fin d'études et enseignant – Université de Biskra}
\end{twocolentry}

\vspace{0.10 cm}
\begin{onecolentry}
    \begin{highlights}
        \item Superviseur pendant le développement du système backend pour Biskra nadifa
        \item Peut attester de mon leadership, coordination d'équipe et compétences backend
        \item Email: \href{mailto:ramdhani@example.com}{mohamed.ramdani@univ-biskra.dz}
        \item Lettre: \href{https://drive.google.com/file/d/1tZtOGG5Xhnyz7CbVRTDBr7Lwauz9Mgk-/view?usp=sharing}{Google Drive}
    \end{highlights}
\end{onecolentry}





    
    \section{Compétences Techniques}

\begin{onecolentry}
    \textbf{Langages:} Java, Python, C, JavaScript, TypeScript
\end{onecolentry}

\begin{onecolentry}
    \textbf{Frameworks:} Spring Boot, Express.js, JavaFX, Android SDK
\end{onecolentry}

\begin{onecolentry}
    \textbf{Bases de données:} MySQL, PostgreSQL, Redis
\end{onecolentry}

\begin{onecolentry}
    \textbf{Outils:} IntelliJ IDEA, Android Studio, Git, Docker, MySQL Workbench
\end{onecolentry}

\begin{onecolentry}
    \textbf{Plateformes cloud:} Render, Aiven, DigitalOcean
\end{onecolentry}

\begin{onecolentry}
    \textbf{Concepts:} API REST, HATEOAS, Spring Security, AOP, CI/CD, UML, BPMN
\end{onecolentry}

\begin{onecolentry}
    \textbf{Autres:} Swagger/OpenAPI, Power BI, CAFYB CRM, Camunda
\end{onecolentry}

\section{Langues}

\begin{onecolentry}
    \textbf{Arabe:} Langue maternelle \\
    \textbf{Anglais:} C1 – Certifié CEIL \\
    \textbf{Français:} B2 – Certifié TCF \\
    \textbf{Allemand:} A2 (Apprentissage actif) \\
\end{onecolentry}
    

\end{document}

